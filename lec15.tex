\documentclass{beamer}
\mode<presentation>
{
\usepackage{external/dis-template}
}
\usepackage{listings}
\usepackage{textcomp}
\usepackage{svg}

\definecolor{comments}{HTML}{50c878}
\lstset{language=C++,
  basicstyle=\ttfamily,
  keywordstyle=\color{blue}\ttfamily,
  stringstyle=\color{red}\ttfamily,
  commentstyle=\color{comments}\ttfamily,
  breaklines=true
}

\graphicspath{{images/}} % TODO: eliminate this hack, necessary because scons builds at repository root

%---------------------------------------------------------------------
\titlepageinit{15}{Firmware Optimization and Engineering}{13 October 2016}
%---------------------------------------------------------------------
\begin{document}
%---------------------------------------------------------------------
\begin{frame}
\titlepage
\setcounter{tocdepth}{1}
\tableofcontents
\end{frame}

%---------------------------------------------------------------------
\section{Optimization} % [?? mins]
%---------------------------------------------------------------------
\begin{frame}
\centering \huge Optimization
\end{frame}


\begin{frame}
\frametitle{Recap}
We brought up the basic hardware and firmware last time \\
\begin{itemize}
  \item Line camera
  \item Servo
  \item Line detection and controller loop
\end{itemize}
\hfill \break
But it's slow
\begin{itemize}
  \item For a fast car: will lose the line by the time steering updates
\end{itemize}
\hfill \break
\textbf{How do we optimize?}
\end{frame}


\begin{frame}
\frametitle{Optimizing}
How much should we optimize? \\
Or, what other limiting factors are there?

\visible<2-> {
\begin{itemize}
  \item Physical
  \visible<3-> {
    \begin{itemize}
      \item Camera exposure / integration time (50ms)
      \item Servo mechanical response (160ms / 60deg)
    \end{itemize}
  }
  \item Electrical
  \visible<3-> {
    \begin{itemize}
      \item Servo update rate (20ms)
    \end{itemize}
  }
\end{itemize}
}
\visible<4-> {
Just for fun, what are some ways around these?
}
\end{frame}


\begin{frame}
\frametitle{Measuring}
How can we measure the update rate? What are some trade-offs? \\

\visible<2-> {
\begin{itemize}
  \item \texttt{printf} and timers
  \item Logic analyzer / oscilloscope on external IO, like \texttt{Serial}
  \item Dedicated timing signals on external IO
\end{itemize}
}

What takes the most amount of time?
\end{frame}


\begin{frame}[fragile]
\frametitle{Overlapped / pipelined operations}
How can we optimize this?
{\tiny \begin{lstlisting}
camera.read_frame(data);  // discard to start integration
wait_ms(50);
camera.read_frame(data);  // actually read the frame
float steering = controller.update(line_detect(data));  // intensive compute
servo.write(steering);
float temperature = read_temperature();  // slow I2C I/O operations
\end{lstlisting}
}
\end{frame}


\begin{frame}[fragile]
\frametitle{Overlapped / pipelined operations}
How can we optimize this?
{\tiny \begin{lstlisting}
camera.read_frame(data);  // discard to start integration
wait_ms(50);
camera.read_frame(data);  // actually read the frame
float steering = controller.update(line_detect(data));  // intensive compute
servo.write(steering);
float temperature = read_temperature();  // slow I2C I/O operations
\end{lstlisting}
}
\hfill \break
Overlap temperature read and line detect with camera integration
{\tiny \begin{lstlisting}
timer.reset();
camera.read_frame(data);  // actually read the frame
float steering = controller.update(line_detect(data)); // intensive compute
servo.write(steering);
float temperature = read_temperature();  // slow I2C I/O operations
while (timer.read_ms() < 50) ;  // block until 50ms elapsed
\end{lstlisting}
}

What are the limits of this technique?
\visible<2-> {\begin{itemize}
  \item Only works when blocking (\texttt{wait(...)}) happens in foreground... \\
  \item What if it's part of a library function, like \texttt{printf}?
\end{itemize}}
\end{frame}


\begin{frame}[fragile]
\frametitle{Threads abstraction}
Manually sequencing operations is \textit{hard}. \\
Why not have the computer handle concurrency with threads?
\begin{columns}[t]
\column{0.5\textwidth}
{\tiny \begin{lstlisting}
void camera_thread() {
  float data[128];
  camera.read_frame(data);
  // intensive compute
  float steering = controller.update(line_detect(data));
  servo.write(steering);
  // explicit yield while integrating
  Thread::wait_ms(50);
}
\end{lstlisting}
}
\column{0.5\textwidth}
{\tiny \begin{lstlisting}
void temperature_thread() {
  // slow I2C I/O operations
  float temperature = read_temperature();
  // ... do something with temperature ...
  // limit update rate with explicit yield
  Thread::wait_ms(50);
}
\end{lstlisting}
}
\end{columns}
\textit{What could \textbf{possibly} go wrong?}
\visible<2->{
  \begin{itemize}
    \item Data synchronization between threads
    \item Hidden limits (jiffy interval, thread switch latency, overhead)
    \item Non-hard-realtime (timing not guaranteed) RTOS
    \item Deadlock, starvation, livelock, ...
  \end{itemize}
\textit{An RTOS is a powerful tool, but comes with caveats. Use with care}
}
\end{frame}


\begin{frame}
\frametitle{DMA architectures}
Even with RTOS / overlapping, most bulk IO operations are blocking. What do? \\
\hfill \break
\visible<2-> {
DMA: \textbf{D}irect \textbf{M}emory \textbf{A}ccess
\begin{itemize}
  \item Automatically transfer blocks of data between memory or peripherals
  \item Operates without CPU intervention, "fire and forget"
  \item Typically: set up transfer (src, dst, length), then start
\end{itemize}
\hfill \break
... but mbed support for DMA is lacking and nonstandard.
\begin{itemize}
  \item If you need this, you will have to register poke
  \item Details likely in chip datasheet / reference manual
\end{itemize}
}
\end{frame}


\begin{frame}[fragile]
\frametitle{Compute optimizations}
What if we didn't have a linear thermistor or FPU?
{\tiny \begin{lstlisting}
void volts_to_temperature(float volts) {
  const float R_REF = 10000;  // 10k thermistor
  const float BETA = 3428, R_INF = R_REF * exp(-BETA / 298.15);
  const float V_REF = 3.3;  // high voltage reference
  float resistance = (R_REF * V_REF / volts) - R_REF;
  float temp = BETA / log((resistance) / R_INF);
  return temp - 273.15;  // degK to degC
}
\end{lstlisting}
}
This is really, really expensive. \\
\hfill \\
How can this be optimized?
\visible<2->{
  \begin{itemize}
    \item Replace with fixed point operations
    \item Lookup tables: precompute values before runtime and store in flash
    \begin{itemize}
      \item Will need to discretize input, for example: \texttt{volts\_to\_temperature(adc.read\_u16())}
    \end{itemize}
  \end{itemize}
}
\end{frame}

%---------------------------------------------------------------------
\section{Software Engineering} % [?? mins]
%---------------------------------------------------------------------


%---------------------------------------------------------------------
\section{Summary} % [?? mins]
%---------------------------------------------------------------------

\begin{frame}
\frametitle{Summary}
\begin{itemize}
  \item Optimizations exist if your application is timing sensitive
  \begin{itemize}
    \item Know what to optimize to: no more, no less
    \item I/O probably is the most time consuming operation
  \end{itemize}
  \item Write good code now so you don't kick yourself later
  \begin{itemize}
    \item Understand what your code is doing
    \item Write clear, concise, readable code
    \item Guidelines may be helpful for safety-critical applications
  \end{itemize}  
\end{itemize}
\end{frame}

\end{document}
